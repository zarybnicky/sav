% Created 2020-12-19 Sat 02:57
% Intended LaTeX compiler: pdflatex
\documentclass[11pt]{article}
\usepackage[utf8x]{inputenc}
\usepackage[T1]{fontenc}
\usepackage{graphicx}
\usepackage{grffile}
\usepackage{longtable}
\usepackage{wrapfig}
\usepackage{rotating}
\usepackage[normalem]{ulem}
\usepackage{amsmath}
\usepackage{textcomp}
\usepackage{amssymb}
\usepackage{capt-of}
\usepackage{hyperref}
\usepackage{minted}
\author{Jakub Zárybnický <xzaryb00@stud.fit.vutbr.cz>}
\date{}
\title{LiquidHaskell: Refinement Types for Haskell}
\hypersetup{
 pdfauthor={Jakub Zárybnický <xzaryb00@stud.fit.vutbr.cz>},
 pdftitle={LiquidHaskell: Refinement Types for Haskell},
 pdfkeywords={},
 pdfsubject={},
 pdfcreator={Emacs 28.0.50 (Org mode 9.3)}, 
 pdflang={English}}
\begin{document}

\maketitle

\section{Introduction}
\label{sec:orgf39f30d}

LiquidHaskell \footnote{\url{https://ucsd-progsys.github.io/liquidhaskell-blog/}} is an actively-developed SMT-based type-checker that adds
\emph{Logically Qualified Data Types} (abbreviated to \emph{Liquid Types}), a variant of
refinement types, to the Haskell type system. It allows the programmer to add
\emph{refinements} to values and data types, while retaining the type inference part of
the Hindley-Milner type system and thus not requiring annotations across the
entire program. Its main limitation is in the kind of properties that it can
prove---only decidable formulas solvable by SMT solvers can be used.

A part of its initial development has been a successful application to many
essential Haskell libraries, including \texttt{base}, \texttt{containers}, \texttt{vector}, and \texttt{bytestring},
proving data type invariants, function totality and termination, or safe access
to indexed structures.

An example of what its syntax looks like can be seen below, where the
prototypical \emph{partial function} \texttt{head}, which is the scarecrow of introductory
Haskell classes---calling it with an empty list will crash the program with a
rather unhelpful error message, a direct contradiction to the type-safety
guarantees brought up by proponents of Haskell.

The Liquid type signature below replaces the plain Haskell type signature, and
asks LiquidHaskell to ensure that the input list is not empty---without changing
the type signature for e.g. external users of a library who may not be using
LiquidHaskell at all.

\begin{minted}[]{haskell}
{-@ head :: {v:[a] | 0 < length v} -> a @-}
head (x:_)  = x

{-@ take :: i:Nat -> {xs:[a] | length xs >= i}
                  -> {ys:[a] | length ys = length xs - i} @-}
take 0 [] = []
take i (x:xs) = if i == 0 then N else (x:take (i-1) xs)
take i [] = impossible "Out of bounds indexing!"
\end{minted}

\subsection{Installation}
\label{sec:orgebd933f}

Until recently, LiquidHaskell has been a separate command-line tool apart from
the Haskell compiler, complicating the workflow especially for larger-scale
projects despite having several editor-specific plugins, and did not really
achieve wide-spread use.

In the summer of 2020, however, there has been initiative to better integrate
the tool with the GHC Haskell compiler via a rather recent addition to GHC which
makes it extensible via \emph{compiler plugins}. This makes adding LiquidHaskell to a
project as easy as adding two lines to the Cabal package specification file, and
also removing the need for tool-specific editor support, which may encourage its
future use. Given that this capability has been just made available in the
latest release, it is also the way in which I tried out the tool.

To use LiquidHaskell, you will need the Haskell compiler GHC, and Z3 (or any
other SMTLIB-compatible SMT solver) on your system. Stack \footnote{\url{https://docs.haskellstack.org/en/stable/README/}} is currently
perhaps the easiest way of obtaining GHC, and Z3 should be available via
the usual software installation route of your system. After that, using
LiquidHaskell should be as easy as adding \texttt{liquidhaskell} as a dependency, and
specifying \texttt{\{-\# OPTIONS -fplugin=LiquidHaskell \#-\}} in any file to be checked by
LiquidHaskell.

\section{Theory}
\label{sec:orgde20dfc}
LiquidHaskell brings refinement types to Haskell. The combination of the
Hindley-Milner system and refinement types has previously been successfully
studied in OCaml \footnote{\url{http://goto.ucsd.edu/\~rjhala/liquid/liquid\_types.pdf}} where it has greatly reduced the amount of manual
program annotation required for program verification. While there are some
complications that arise from translating the technique from a call-by-value
language like OCaml to a call-by-need language like Haskell---that LiquidHaskell
has solved by tying type verification to termination checking---the general
technique still applies.

LiquidHaskell works by letting GHC translate the program into Core \footnote{\url{https://gitlab.haskell.org/ghc/ghc/-/wikis/commentary/compiler/core-syn-type}}, a
small, explicitly-typed version of System F. It then takes this output and the
result of parsing the type refinement annotations, and translates them into Horn
clause constraints, \emph{verification conditions}, and lets an SMT solver solve them.

Refinement type systems are designed so that the refinement logic is efficiently
decidable, and unlike full dependent types do not allow arbitrary functions in
the refinement logic. LiquidHaskell constrains the annotation language to the
\emph{decidable logic of equality, uninterpreted functions and linear arithmetic},
QF-EUFLIA. This constrains the language of annotations, especially in the use of
functions.

There are several types of annotations available: the most common one are
function signature annotations which have already been mentioned above, that
introduce pre- and post-conditions of the function. The next one is \texttt{data} that
creates refined data types, adding refinements to fields and constructors of
Haskell data types.

\begin{minted}[]{haskell}
{-@
data Map k a <l :: k -> k -> Prop, r :: k -> k -> Prop>
  = Tip
  | Bin (sz    :: Size)
        (key   :: k)
        (value :: a)
        (left  :: Map <l, r> (k <l key>) a)
        (right :: Map <l, r> (k <r key>) a)
@-}
data Map k a = Tip
             | Bin Size k a (Map k a) (Map k a)
\end{minted}

The next one is \texttt{measure}, and the related annotations \texttt{inline} and \texttt{reflect}, that
copy the function definition to the refinement logic. While this may seem to
allow arbitrary functions, they are quite restricted---\texttt{inline} cannot be
recursive. \texttt{measure}, named as the \emph{measure} of a data type, may only recurse on the
constructors of the data type.

\begin{minted}[]{haskell}
data Heap a = Empty | Node { left :: Heap a, el :: a, right :: Heap a }

{-@ measure heapSize @-}
heapSize :: Heap a -> Int
heapSize Empty = 0
heapSize (Node l _ r) = 1 + heapSize l + heapSize r
\end{minted}

The last two (non-deprecated) annotations are \texttt{type} and \texttt{assume}, where \texttt{type}
introduces a type alias that may contain pre- or post-conditions, and \texttt{assume}
adds an unchecked fact into the logic. An example of a type alias can be seen
below.

\begin{minted}[]{haskell}
{-@ type HeapN a N = {h:Heap a | heapSize h = N} @-}

{-@ exampleHeap :: HeapN Int 2 @-}
exampleHeap = Node (Node Empty 1 Empty) 0 Empty
\end{minted}

If, for example, the explicitly named size of the heap was specified badly, we
would get an error message like below. This is the common format of all
LiquidHaskell error messages---as is sometimes said about Haskell, reading error
messages is a skill by itself, as the cause of the error is not always
immediately obvious.

\begin{minted}[]{text}
Liquid Type Mismatch
  .
  The inferred type
    VV : (Lib.Heap [GHC.Types.Int])
  .
  is not a subtype of the required type
    VV : {VV : (Lib.Heap [GHC.Types.Int]) | Lib.heapSize VV == 2}
  .
\end{minted}

\section{Showcase}
\label{sec:orgfd7ba5f}
LiquidHaskell comes with a large array of showcases, tutorials, and
exercises---in this work I will walk through one of these, an implementation of
the functional data structure \emph{Lazy Queues} \footnote{\url{http://www.westpoint.edu/eecs/SiteAssets/SitePages/Faculty\%20Publication\%20Documents/Okasaki/jfp95queue.pdf}}, that is available as a set of
exercises without a publicly available solution \footnote{\url{http://ucsd-progsys.github.io/liquidhaskell-tutorial/Tutorial\_09\_Case\_Study\_Lazy\_Queues.html}}. A lazy queue is a structure
based on two linked lists whose amortized performance of \(O(1)\) of the \emph{insert}
and \emph{remove} operations depends on an invariant being enforced between the two
lists, namely that the first list need to be longer or equal to the second
one---an ideal use-case a program verification tool. We will start with two data
structures, a sized linked list, and the queue:

\begin{minted}[]{haskell}
data SList a = SL { size :: Int, elems :: [a] }

data Queue a = Q { front :: SList a, back  :: SList a }
\end{minted}

To bind the value of the \texttt{size} field to the length of the list \texttt{elems}, we will
need a few more annotations, one to declare the dependency, and one to declare
the type of the field accessor function. We will also need the \emph{measure} function,
to give the SMT solver a way to get from the list to its size. We can also
declare two type aliases that will make our life easier, one for a list of a
specific size, one for a non-empty list:

\begin{minted}[]{haskell}
{-@ measure listSize @-}
listSize :: [a] -> Int
listSize []     = 0
listSize (_:xs) = 1 + listSize xs

{-@ data SList a = SL
  { size  :: Nat
  , elems :: {v:[a] | listSize v = size}
  } @-}
{-@ size :: q:SList a -> {v:Nat | v = size q} @-}

{-@ type SListN a N = {v:SList a | size v = N} @-}
{-@ type NESList a  = {v:SList a | size v > 0} @-}
\end{minted}

Doing the same for the queue will yield us the foundation, only this time we
don't have a size field, but we need to declare the invariant---tell the solver
that \texttt{back} needs to always be smaller or equal to \texttt{front}.

\begin{minted}[]{haskell}
{-@ measure queueSize @-}
queueSize :: Queue a -> Int
queueSize (Q f b) = size f + size b

{-@ type SListLE a N = {v:SList a | size v <= N} @-}
{-@ data Queue a = Q {
  { front :: SList a
  , back  :: SListLE a (size front)
  } @-}

{-@ type QueueN a N = {v:Queue a | queueSize v = N} @-}
{-@ type NEQueue a = {v:Queue a | queueSize v > 0} @-}
\end{minted}

If we now declare a badly-sized list like \texttt{list = SL 1 []} we will get a rather
interesting message. We see that LiquidHaskell has inferred several properties
of the empty list passed to the constructor---properties of its length and its
identity---and that those fail to match with the invariants on the constructor.

\begin{minted}[]{haskell}
Liquid Type Mismatch
  .
  The inferred type
    VV : {v : [a] | Lib.realSize v == 0
                    && len v == 0
                    && len v >= 0
                    && v == ?c}
  .
  is not a subtype of the required type
    VV : {VV : [a] | Lib.realSize VV == 1}
  .
  in the context
    ?c : {?c : [a] | Lib.realSize ?c == 0
                     && len ?c == 0
                     && len ?c >= 0}
\end{minted}

Before starting to work on the queue itself, we need a few functions to
manipulate the sized list, equivalents to the library functions \texttt{head} and \texttt{tail},
perhaps only there as an exercise. We can see quite well how the size of the
list varies in each operation, their pre- and post-conditions.

\begin{minted}[]{haskell}
{-@ nil :: SListN a 0 @-}
nil = SL 0 []

{-@ cons :: a -> xs:SList a -> SListN a {size xs + 1} @-}
cons x (SL n xs) = SL (n+1) (x:xs)

{-@ tl :: xs:NESList a -> SListN a {size xs - 1}  @-}
tl (SL n (_:xs)) = SL (n-1) xs
tl _ = error "empty SList"

{-@ hd :: xs:NESList a -> a @-}
hd (SL _ (x:_)) = x
hd _ = error "empty SList"
\end{minted}

While the exercise is spread over several sections, I will condense the last
definitions into a single block. We need to define the operations over the
queue, \texttt{insert} and \texttt{remove}. In the simple case, we simply need to \emph{cons} or \emph{uncons}
an element on the two lists, but if the invariant would be broken, we need a way
to restore it---and that would be \texttt{makeq} and \texttt{rot}, where \texttt{rot} rotates the \texttt{back} list
and appends it to the \texttt{front} one. Using these two, implementing \texttt{insert} and \texttt{remove}
is trivial:

\begin{minted}[]{haskell}
{-@ makeq :: f:SList a -> b:SListLE a {size f + 1}
          -> QueueN a {size f + size b} @-}
makeq f b
  | size b <= size f = Q f b
  | otherwise        = Q (rot f b nil) nil

{-@ rot :: l:SList a -> r:SListN a {size l + 1} -> a:SList a
         -> SListN a {size l + size r + size a} @-}
rot f b acc
  | size f == 0 = hd b `cons` acc
  | otherwise   = hd f `cons` rot (tl f) (tl b) (hd b `cons` acc)

{-@ insert :: a -> q: Queue a -> QueueN a {queueSize q + 1} @-}
insert e (Q f b) = makeq f (e `cons` b)
{-@ remove :: q:NEQueue a -> (a, QueueN a {queueSize q - 1})  @-}
remove (Q f b) = (hd f, makeq (tl f) b)
\end{minted}

The exercise ends with a final test, to implement \texttt{replicate} and verify that it
works:

\begin{minted}[]{haskell}
{-@ replicate :: n:Nat -> a -> QueueN a n @-}
replicate 0 _ = emp
replicate n x = insert x (replicate (n-1) x)

{-@ okReplicate :: QueueN _ 3 @-}
okReplicate = replicate 3 "Yeah!"  -- accept

{-@ badReplicate :: QueueN _ 3 @-}
badReplicate = replicate 1 "No!"   -- reject
\end{minted}

The last line is rejected with an error very similar to the above one, informing
us of the type mismatch between the value and the type we claim it has, which
also means that the refinement types of the above expressions are well-typed and
work well.

\section{Comparison with Dependent Haskell}
\label{sec:org66dc263}
Bringing fully dependent types to Haskell is a long-term aspiration of several
interest groups \footnote{\url{https://github.com/goldfirere/thesis/}}, which may replace tools like LiquidHaskell and move
Haskell closer in power to languages like Agda or Idris.

However, it is not unrealistic to get somewhat close to the power of refinement
or dependent types by using several GHC extensions to Haskell, using \texttt{DataKinds}
to promote values to the type-level, and \texttt{TypeFamilies} to achieve type-level
functions. Below you can see a snipped of code which has been taken from a
previous Haskell proof-of-concept of mine, an implementation of a type-safe and
verified \emph{Braun heap} data structure. You can see that it is necessary to manually
prove certain properties, adding the data type \texttt{Offset} and prove type equalities
using \texttt{Data.Type.Equality}, due to the lack of support for arithmetic---the last
two lines simply prove that if \(y + z = w\) and \(x = 1 + y + z\), then \(x = 1 + w\)
to the compiler.

\begin{minted}[,fontsize=\footnotesize]{haskell}
data Heap (n :: Nat) a where
  Empty :: Heap 0 a
  Node :: Offset m n -> Heap m a -> a -> Heap n a -> Heap (1 + m + n) a
data Offset m n where
  Even :: Offset n n
  Leaning :: Offset (1 + n) n

merge :: Ord a => Offset m n -> Heap m a -> Heap n a -> Heap (m + n) a
merge Even = mergeEven
merge Leaning = mergeLeaning

mergeEven :: Ord a => Heap n a -> Heap n a -> Heap (n + n) a
mergeEven l@(Node lo ll lx lr) r@(Node _ _ ly _)
  | lx <= ly = Node Leaning r lx (merge lo ll lr)
  | otherwise = let (x, l') = extract l in Node Leaning (replaceMin x r) ly l'
mergeEven _ _ = Empty

mergeLeaning :: Ord a => Heap (1 + n) a -> Heap n a -> Heap (1 + n + n) a
mergeLeaning h Empty = h
mergeLeaning Empty h = h
mergeLeaning l@(Node lo ll lx lr) r@(Node _ rl ly rr)
  | lx <= ly = Node Even r lx (merge lo ll lr)
  | otherwise = case proof r rl rr Refl of
      Refl -> let (x, l') = extract l in Node Even (replaceMin x r) ly l'
  where
    proof :: ((y + z) ~ w) => p x a -> p y a -> p z a -> x :~: (1 + y + z) -> x :~: (1 + w)
    proof _ _ _ Refl = Refl
\end{minted}

Dependent types in Haskell, however, are most closely approximated by the
\texttt{singletons} library, which allows using both values as types, and types as
values - the following is taken from the \texttt{GPLVMHaskell} library, a function that
takes an opaque matrix data type and promotes its dimensions to the type-level
as \texttt{Nat}, and makes them available to a receiver function, perhaps allowing it to
prove it is working with a square matrix.

\begin{minted}[,fontsize=\footnotesize]{haskell}
withMat
  :: Matrix D Double
  -> (forall (x :: Nat) (y :: Nat). (SingI y, SingI x) => DimMatrix D x y Double -> k)
  -> k
withMat m f =
  let (Z  :.  y  :.  x) = extent m
  in
  case toSing (intToNat y) of
    SomeSing (sy :: Sing m) -> withSingI sy $
      case toSing (intToNat x) of
        SomeSing (sx :: Sing n) -> withSingI sx $ f (DimMatrix @D @m @n m)
\end{minted}

\section{Conclusion}
\label{sec:org9e55f7e}
While I personally would not reach for LiquidHaskell in my future projects due
to several remaining ergonomy issues---lack of support for the documentation
generation tool Haddock, or not being able to use GHCJS, the
Haskell-to-JavaScript compiler---it seems to be an advanced tool that helps push
the Haskell guarantees of type-safety and correctness-by-construction even
further, that has already proved itself in a number of security-critical
applications, unlike many other program verification tools that primarily serve
as a research vehicle.
\end{document}